%\documentclass[MTech]{cetlyproj}
\documentclass[BTech]{cetlyproj}
\usepackage{times}
 \usepackage{t1enc}
\usepackage{setspace}
\usepackage{graphicx}
\usepackage{epstopdf}
\usepackage[driverfallback=dvipdfm]{hyperref} % hyperlinks for references.
\usepackage{amsmath} % easier math formulae, align, subequations \ldots

\begin{document}


%%%%%%%%%%%%%%%%%%%%%%%%%%%%%%%%%%%%%%%%%%%%%%%%%%%%%%%%%%%%%%%%%%%%%%
% Title page
% Your project/thesis title here
\title{\LaTeX\ TEMPLATE FOR PROJECT REPORT TO KTU}
% Your name/names here
\author{Name1 (TLY15XXXXX)\\Name2 (TLY15XXXXX)\\Name3 (TLY15XXXXX)\\ \& \\Name4 (TLY15XXXXX)}

\date{MAY 2019}
\department{MECHANICAL ENGINEERING}
\programme{MECHANICAL ENGINEERING}
%\programme{MANUFACTURING SYSTEMS MANAGEMENT}

%\nocite{*}
\maketitle

%%%%%%%%%%%%%%%%%%%%%%%%%%%%%%%%%%%%%%%%%%%%%%%%%%%%%%%%%%%%%%%%%%%%%%
%Declaration
\declaration

\vspace*{0.25in}

I undersigned hereby declare that the project report {\bf "Title of the project" }, submitted for partial fulfillment of the requirements for the award of degree of Master of Technology of the APJ Abdul Kalam Technological University, Kerala is a bonafide work done by me under supervision of {\bf Name of supervisor(s)} . This submission represents my ideas in my own words and where ideas or words of others have been included, I have adequately and accurately cited and referenced the original sources. I also declare that I have adhered to ethics of academic honesty and integrity and have not misrepresented or fabricated any data or idea or fact or source in my submission. I understand that any
violation of the above will be a cause for disciplinary action by the institute and/or the University and can also evoke penal action from the sources which have thus not been properly cited or from whom proper permission has not been obtained. This report has not been previously formed the basis for the award of any degree, diploma or similar title
of any other University.\par
\vspace*{0.2in}
\begin{singlespacing}
	\hspace*{-0.25in}
	\parbox{2.5in}{
		\vspace{3.2in}
		\noindent Place \\
		\noindent Date \\
	} 
	\hspace*{1.0in}
	\parbox{2.5in}{
		\noindent
		\noindent Name1\\
		\noindent TLY15XXXXX
		\\
		\\ \\
		\noindent
		\noindent Name2\\
		\noindent TLY15XXXXX
		\\
		\\ \\
		\noindent
		\noindent Name3\\
		\noindent TLY15XXXXX
		\\
		\\ \\
		\noindent
		\noindent Name4\\
		\noindent TLY15XXXXX
	}  
	%% End co guide
\end{singlespacing}
%%%%%%%%%%%%%%%%%%%%%%%%%%%%%%%%%%%%%%%%%%%%%%%%%%%%%%%%%%%%%%%%%%%%%%
% Certificate
\certificate

\vspace*{0.25in}

\noindent This is to certify that the report entitled {\large \bf TITLE OF THE PROJECT} submitted by {\bf NAME1, NAME2, NAME3} and {\bf NAME4} to the APJ Abdul Kalam Technological University in partial fulfillment of the requirements for the award of the Degree of Master of Technology in ( stream \& branch) is a bonafide record of the project work carried out by him/her under my/our guidance and supervision. This report in any form has not been submitted to any
other University or Institute for any purpose.

\vspace*{1.0in}

\begin{singlespacing}
\hspace*{-0.25in}
\parbox{2.5in}{
\noindent {\bf Prof.~1} \\
\noindent Project Guide \\ 
\noindent Assistant Professor \\
\noindent Dept. of Mechanical Engineering\\
\noindent College of Engineering Thalassery \\
} 
\hspace*{0.7in} 
%% HOD
\parbox{2.5in}{
\noindent {\bf Prof.~2} \\
\noindent Professor \& HOD \\ 
\noindent Dept. of Mechanical Engineering\\
\noindent College of Engineering Thalassery \\
}  
%% End HOD
\end{singlespacing}
\vspace*{0.25in}

\noindent Place: Thalassery\\
Date: 


%%%%%%%%%%%%%%%%%%%%%%%%%%%%%%%%%%%%%%%%%%%%%%%%%%%%%%%%%%%%%%%%%%%%%%
% Acknowledgements
\acknowledgements

Thanks to all those who made \TeX\ and \LaTeX\ what is today.

%%%%%%%%%%%%%%%%%%%%%%%%%%%%%%%%%%%%%%%%%%%%%%%%%%%%%%%%%%%%%%%%%%%%%%
% Abstract

\abstract


\vspace*{24pt}

\noindent A \LaTeX\ class along with a simple template report are
provided here.  These can be used to easily write a report suitable
for submission to KTU at College of Engineering Thalassery.  The class provides options to format PhD thesis, M.Tech.\ and B.Tech.\ project reports. 

The formatting is as (as far as the author is aware) per the current
guidelines by KTU. This \href{https://www.ktu.edu.in/data/M.TechProjectReportGuidlines.pdf}{Sample Report} is used as a reference for preparing this template\\

\noindent KEYWORDS: \hspace*{0.5em} \parbox[t]{4.4in}{\LaTeX ; Thesis;
Style files; Format.}

\pagebreak

%%%%%%%%%%%%%%%%%%%%%%%%%%%%%%%%%%%%%%%%%%%%%%%%%%%%%%%%%%%%%%%%%
% Table of contents etc.

\begin{singlespace}
\tableofcontents
\thispagestyle{empty}

\listoftables
\addcontentsline{toc}{chapter}{LIST OF TABLES}
\listoffigures
\addcontentsline{toc}{chapter}{LIST OF FIGURES}
\end{singlespace}


%%%%%%%%%%%%%%%%%%%%%%%%%%%%%%%%%%%%%%%%%%%%%%%%%%%%%%%%%%%%%%%%%%%%%%
% Abbreviations
\abbreviations

\noindent 
\begin{tabbing}
xxxxxxxxxxx \= xxxxxxxxxxxxxxxxxxxxxxxxxxxxxxxxxxxxxxxxxxxxxxxx \kill
\textbf{COET}   \> College of Engineering Thalassery \\
\textbf{RTFM} \> Read the Fine Manual \\
\end{tabbing}

\pagebreak

%%%%%%%%%%%%%%%%%%%%%%%%%%%%%%%%%%%%%%%%%%%%%%%%%%%%%%%%%%%%%%%%%%%%%%
% Notation

\chapter*{\centerline{NOTATION}}
\addcontentsline{toc}{chapter}{NOTATION}

\begin{singlespace}
\begin{tabbing}
xxxxxxxxxxx \= xxxxxxxxxxxxxxxxxxxxxxxxxxxxxxxxxxxxxxxxxxxxxxxx \kill
\textbf{$r$}  \> Radius, $m$ \\
\textbf{$\alpha$}  \> Angle of thesis in degrees \\
\textbf{$\beta$}   \> Flight path in degrees \\
\end{tabbing}
\end{singlespace}

\pagebreak
\clearpage

% The main text will follow from this point so set the page numbering
% to arabic from here on.
\pagenumbering{arabic}


%%%%%%%%%%%%%%%%%%%%%%%%%%%%%%%%%%%%%%%%%%%%%%%%%%
% Introduction.

\chapter{INTRODUCTION}
\label{chap:intro}

This document provides a simple template of how the provided
\verb+cetlyproj.cls+ \LaTeX\ class is to be used.  Also provided are
several useful tips to do various things that might be of use when you
write your report/thesis.\par

\section{Instructions}
\begin{itemize}
	\item Install TexLive, MiKTeX or MacTex:
	\item General installation \href{http://www.tug.org/texlive/acquire-netinstall.html}{instructions}
	\item Ubuntu: \verb+sudo apt-get install texlive-full+
	\item Edit report.tex file using any text editor.
	\item TeXStudio preferred: \href{http://www.texstudio.org/}{Download here}
	\item Install on Ubuntu: \verb+sudo apt-get install texstudio+
	\item You can use any other \LaTeX installation of your choice, but \verb+TexLive Full+ install will guarantee that all required packages are installed.
	\item In case you are having trouble using \LaTeX on your machine, simply create an account at \href{https://www.sharelatex.com/}{ShareLaTeX} and create a new project and upload a zip of this project.
\end{itemize}
\section{Package Options}

Use this template as a basic template to format your report/thesis.  The
\verb+cetlyproj+ class can be used by simply using something like this:
\begin{verbatim}
\documentclass[BTech]{cetlyproj}  
\end{verbatim}

To change the title page for different degrees just change the option
from \verb+PhD+ to one of \verb+MTech+ or \verb+BTech+. The title page formatting really depends on how large or small your thesis title is.  Consequently it might require some hand tuning.  Edit your version of \verb+cetlyproj.cls+ suitably to do this.I recommend that this be done once your title is final.

Once again the title page may require some small amount of fine
tuning.  This is again easily done by editing the class file.

This sample file uses the \verb+hyperref+ package that makes all
labels and references clickable in PDF files.  These are very useful when reading the document online and do not affect the output when the files are printed.


\section{Example Figures and tables}

Fig.~\ref{fig:logos} shows a simple figure for illustration along with
a long caption.  The formatting of the caption text is automatically
single spaced and indented.  Table~\ref{tab:example} shows a sample
table with the caption placed correctly.  The caption for this should
always be placed before the table as shown in the example.


\begin{figure}[htpb]
  \begin{center}
    \resizebox{50mm}{!} {\includegraphics *{cetly.eps}}
    \resizebox{50mm}{!} {\includegraphics *{cetly.eps}}
    \caption {Two CoET logos in a row.  This is also an
      illustration of a very long figure caption that wraps around two
      two lines.  Notice that the caption is single-spaced.}
  \label{fig:logos}
  \end{center}
\end{figure}

\begin{table}[htbp]
  \caption{A sample table with a table caption placed
    appropriately. This caption is also very long and is
    single-spaced.  Also notice how the text is aligned.}
  \begin{center}
  \begin{tabular}[c]{|c|r|} \hline
    $x$ & $x^2$ \\ \hline
    1  &  1   \\
    2  &  4  \\
    3  &  9  \\
    4  &  16  \\
    5  &  25  \\
    6  &  36  \\
    7  &  49  \\
    8  &  64  \\ \hline
  \end{tabular}
  \label{tab:example}
  \end{center}
\end{table}

\section{Bibliography with BIB\TeX}

It is strongly recommend that you use BIB\TeX\ to automatically generate
your bibliography.  It makes managing your references much easier.  It
is an excellent way to organize your references and reuse them.  You
can use one set of entries for your references and cite them in your
thesis, papers and reports.  If you haven't used it anytime before
please invest some time learning how to use it.  

A simple example BIB\TeX\ file along in this directory called \verb+refs.bib+. The \verb+cetlyproj.cls+ class package which
is used in this thesis uses the \verb+natbib+ package to format the references along with a customized bibliography style provided as the \verb+apjktu.bst+ file in the directory containing \verb+report.tex+.  Documentation for the \verb+natbib+ package should be available in your distribution of \LaTeX.  Basically, to cite the author along with the author name and year use \verb+\cite{key}+ where \verb+key+ is the citation key for your bibliography entry.  You can also use \verb+\citet{key}+ to get the same effect.  To make the citation without the author name in the main text but inside the parenthesis use \verb+\citep{key}+.  The following paragraph shows how citations can be used in text effectively.

More information on BIB\TeX\ is available in the book by
\cite{lamport:86}.  There are many references~\citep{lamport:86,latexcompanion} that explain how to use
BIB\TeX.  Read the \verb+natbib+ package documentation for more
details on how to cite things differently.

Here are other references for example.\citet{einstein:905} shows another example to cite an article.\citet{latexcompanion} illustrates a book with multiple authors.  Python~\citep{py:python} is a programming language and is
cited here to show how to cite something that is best identified with
a URL.

\section{Other useful \LaTeX\ packages}

The following packages might be useful when writing your thesis.

\begin{itemize}  
\item It is very useful to include line numbers in your document.
  That way, it is very easy for people to suggest corrections to your
  text.  I recommend the use of the \texttt{lineno} package for this
  purpose.  This is not a standard package but can be obtained on the
  internet.  The directory containing this file should contain a
  lineno directory that includes the package along with documentation
  for it.

\item The \texttt{listings} package should be available with your
  distribution of \LaTeX.  This package is very useful when one needs
  to list source code or pseudo-code.

\item For special figure captions the \texttt{ccaption} package may be
  useful.  This is specially useful if one has a figure that spans
  more than two pages and you need to use the same figure number.

\item The notation page can be entered manually or automatically
  generated using the \texttt{nomencl} package.

\end{itemize}

More details on how to use these specific packages are available along
with the documentation of the respective packages.

%%%%%%%%%%%%%%%%%%%%%%%%%%%%%%%%%%%%%%%%%%%%%%%%%%%%%%%%%%%%
% Appendices.

\appendix

\chapter{A SAMPLE APPENDIX}

Just put in text as you would into any chapter with sections and
whatnot.  Thats the end of it.

%%%%%%%%%%%%%%%%%%%%%%%%%%%%%%%%%%%%%%%%%%%%%%%%%%%%%%%%%%%%
% Bibliography.

\begin{singlespace}
  \bibliography{refs}
\end{singlespace}


%%%%%%%%%%%%%%%%%%%%%%%%%%%%%%%%%%%%%%%%%%%%%%%%%%%%%%%%%%%%
% List of papers

\listofpapers

\begin{enumerate}  
\item Authors....  \newblock
 Title...
  \newblock {\em Journal}, Volume,
  Page, (year).
\end{enumerate}  

\end{document}
